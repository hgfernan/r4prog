\documentclass[a4paper,10pt]{article}
\usepackage{latexsym}
\usepackage[brazilian]{babel}
\usepackage{lmodern}
\usepackage[utf8]{inputenc}
\usepackage[T1]{fontenc}
\usepackage{longtable}
\usepackage{graphicx}

\usepackage{hyperref}
\usepackage{amsmath}  % for equation environment
\usepackage{enumitem} % nolistsep to reduce list spacing

%opening
\title{Apresentando R através do COVID-19: apontamentos para palestra}
\author{Hilton Fernandes}

\begin{document}

\maketitle

\begin{abstract}
Appointments for an introductory lecture to R using COVID as a motivation.
\end{abstract}

\tableofcontents

\section{Motivação}

\section{Coletando dados de COVID-19 no Brasil}

Há na Internet muitas fontes para obtenção de dados sobre a epidemia. No fundo, 
todas elas derivam de dados fornecidos pelo Ministério da Saúde do Brasil que, 
por sua vez, concentra dados das secretarias de saúde dos governos estaduais. 

Devido a questões delicadas, que não serão comentadas aqui, em algum momento o 
Ministério da Saúde do Brasil deixou de coletar e disponibilizar esses dados, 
que foram então concentrados pelo consórcio de mídia:

\begin{enumerate}
    \item baixar os dados do {\tt github};
    \item examinar o pacote de dados: 
	\begin{enumerate}
	    \item {\tt View()} para examinar dados;
	    \item Contagem de linhas;
	    \item Contagem de colunas;
	    \item listagem de estruturas.
	\end{enumerate}

    \item selecionar apenas os dados do Brasil;
    \item selecionar apenas os dados do começo da pandemia;
    \item curvas para variáveis -- foco em valores acumulados, por reduzirem 
	oscilações.
\end{enumerate}


\section{Aproximação por regressão linear}

\section{Modelo geral de epidemias}
\section{\texorpdfstring{A transformação {\em log-log}}
			{A transformação log-log}
	}
\section{Aproximação de um modelo exponencial aos dados de COVID-19 do Brasil}
\section{Interpretação do modelo exponencial aos dados de COVID-19 do Brasil}
\section{\texorpdfstring{Apresentação dos recursos do $R$ que foram usados}
			{Apresentação dos recursos do R que foram usados}
	}


\end{document}
