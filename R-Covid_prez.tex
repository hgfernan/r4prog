\documentclass[a4paper,10pt]{beamer}
\usepackage{latexsym}
\usepackage[brazilian]{babel}
\usepackage{lmodern}
\usepackage[utf8]{inputenc}
\usepackage[T1]{fontenc}
\usepackage{longtable}
\usepackage{graphicx}

\usepackage{hyperref}
\usepackage{amsmath}  % for equation environment
% \usepackage{enumitem} % nolistsep to reduce list spacing
\usepackage{default}

\begin{document}

\begin{frame}
  \maketitle{Usando {\em R} para entender a COVID-19}
  
\end{frame}

\section{Introdução}

\begin{frame}{Agenda}
 \tableofcontents
\end{frame}

\begin{frame}{O impacto da COVID-19}
  \begin{itemize}
      \item O impacto da pandemia COVID-19 talvez seja ainda maior do que de 
	  outras pandemias;
      \item Talvez seja maior do que aquele da Peste Negra da Idade Média;
      \item Talvez seja ainda maior do que a pandemia da chamada 
	  {\em Gripe Espanhola} no começo do século passado.
      \item A razão é que hoje o mundo está muito mais conectado.
      \item O {\em outsourcing} -- uso de fornecedores externos de serviços e
	  bens materiais -- está muito 
	  mais amplo do que em qualquer época da humanidade
  \end{itemize}

\end{frame}

\begin{frame}{Metas para esta apresentação}
  \begin{itemize}
      \item A pandemia da COVID-19 tem efeitos muito complexos na sociedade.
      \item Uma primeira forma de abordá-la é entender a velocidade de sua 
	  propagação.
      \item Tanto para compreender a história de sua evolução;
      \item Como para tentar prever seu futuro impacto, para que se possam
	  elaborar políticas mais eficazes para contê-la.
	  
      \item Neste pequeno vídeo falaremos de alguns modelos matemáticos simples
	  que podem ajudar a compreender o passado e a prever o futuro.
      \item Para isso usaremos o software chamado $R$, que tem sido considerado
	  um dos mais completos para estatística.
      \item O plano, então, é tanto refletir sobre COVID-19, quanto aprender 
	  a usar $R$.
  \end{itemize}

\end{frame}

\section{Coletando dados de COVID-19}

\begin{frame}{Fontes de dados da COVID-19 no Brasil}
    \begin{itemize}
	\item No Brasil, os dados de contaminação da COVID-19 tem sido coletados
	    pelas prefeituras, concentrados pelos governadores dos estados 
	    e finalmente agregados pelo Ministério da Saúde.
	\item Mas, na prática, os dados do Brail tem sido obtidos a partir da 
	    página de instituto especializado da Universidade John Hopkins, 
 	    dos EUA.
	\item Esta página coleta dados de todo o mundo, e tem recursos para
	    comparar diferentes progressões das epidemias no mundo.
	 \item E tem ferramentas de software para realizar algumas operações 
	     matemáticas sobre os dados.
    \end{itemize}

\end{frame}


\begin{frame}{A página de onde são coletados os dados}
  \begin{itemize}
      \item A
  \end{itemize}

\end{frame}

\section{Aproximação por regressão linear}

\begin{frame}{A regressão linear para aproximar funções a dados empíricos}
    \begin{itemize}
	\item a
    \end{itemize}

\end{frame}

\begin{frame}{Falha da regressão linear para aproximar os dados de COVID-19}
    \begin{itemize}
	\item a
    \end{itemize}

\end{frame}


\section{Modelo geral de epidemias}

\begin{frame}{Como é realizado o contágio, em uma epidemia ?}
    \begin{itemize}
	\item a
    \end{itemize}

\end{frame}

\section{A transformação {\em log-log}}

\begin{frame}{Sobre a transformação {\em log-log}}
    \begin{itemize}
	\item a
    \end{itemize}

\end{frame}

\begin{frame}{Por que a transformação {\em log-log} ajuda neste caso ?}
    \begin{itemize}
	\item a
    \end{itemize}

\end{frame}

\section{Aproximação de um modelo exponencial aos dados de COVID-19 do Brasil}

\begin{frame}{O novo modelo, sob {\em log-log}}
    \begin{itemize}
	\item a
    \end{itemize}

\end{frame}

\begin{frame}{Inversão da transformação {\em log-log}}
    \begin{itemize}
	\item a
    \end{itemize}

\end{frame}

\section{Interpretação do modelo exponencial aos dados de COVID-19 do Brasil}

\begin{frame}{Avaliação da qualidade do modelo aproximado}
    \begin{itemize}
	\item a
    \end{itemize}

\end{frame}

\begin{frame}{Interpolação versus extrapolação}
    \begin{itemize}
	\item a
    \end{itemize}

\end{frame}

\begin{frame}{Algumas previsões}
    \begin{itemize}
	\item a
    \end{itemize}

\end{frame}

\begin{frame}{Limites desta análise}
    \begin{itemize}
	\item a
    \end{itemize}

\end{frame}

\section{Apresentação dos recursos do 
	  \texorpdfstring{$R$}{R} 
	  que foram usados}

\begin{frame}{Depois do uso das ferramentas, sua compreensão}
    \begin{itemize}
	\item a
    \end{itemize}

\end{frame}

\subsection{Leitura de dados}

\begin{frame}{Arquivos de dados e {\em dataframes}}
    \begin{itemize}
	\item a
    \end{itemize}

\end{frame}

\begin{frame}{Variáveis obtidas do conjunto de dados}
    \begin{itemize}
	\item a
    \end{itemize}

\end{frame}

\subsection{Transformação de variáveis}

\begin{frame}{Variáveis transformadas}
    \begin{itemize}
	\item a
    \end{itemize}

\end{frame}

\begin{frame}{Em $R$ (quase) tudo é uma lista}
    \begin{itemize}
	\item a
    \end{itemize}

\end{frame}

\subsection{Recursos matemáticos disponíveis}

\begin{frame}{O $R$ é uma biblioteca de matemática}
    \begin{itemize}
	\item a
    \end{itemize}

\end{frame}

\begin{frame}{Funções e algoritmos estatísticos}
    \begin{itemize}
	\item a
    \end{itemize}

\end{frame}

\subsection{Análise do resultado das regressões}

\begin{frame}{Dois parâmetros importantes}
    \begin{itemize}
	\item a
    \end{itemize}

\end{frame}

\subsection{Recursos de representação gráfica dos dados}

\begin{frame}{Representação de funções de uma varíável}
    \begin{itemize}
	\item a
    \end{itemize}

\end{frame}

\end{document}
